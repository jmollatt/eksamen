% Options for packages loaded elsewhere
\PassOptionsToPackage{unicode}{hyperref}
\PassOptionsToPackage{hyphens}{url}
%
\documentclass[
]{article}
\usepackage{amsmath,amssymb}
\usepackage{lmodern}
\usepackage{ifxetex,ifluatex}
\ifnum 0\ifxetex 1\fi\ifluatex 1\fi=0 % if pdftex
  \usepackage[T1]{fontenc}
  \usepackage[utf8]{inputenc}
  \usepackage{textcomp} % provide euro and other symbols
\else % if luatex or xetex
  \usepackage{unicode-math}
  \defaultfontfeatures{Scale=MatchLowercase}
  \defaultfontfeatures[\rmfamily]{Ligatures=TeX,Scale=1}
\fi
% Use upquote if available, for straight quotes in verbatim environments
\IfFileExists{upquote.sty}{\usepackage{upquote}}{}
\IfFileExists{microtype.sty}{% use microtype if available
  \usepackage[]{microtype}
  \UseMicrotypeSet[protrusion]{basicmath} % disable protrusion for tt fonts
}{}
\makeatletter
\@ifundefined{KOMAClassName}{% if non-KOMA class
  \IfFileExists{parskip.sty}{%
    \usepackage{parskip}
  }{% else
    \setlength{\parindent}{0pt}
    \setlength{\parskip}{6pt plus 2pt minus 1pt}}
}{% if KOMA class
  \KOMAoptions{parskip=half}}
\makeatother
\usepackage{xcolor}
\IfFileExists{xurl.sty}{\usepackage{xurl}}{} % add URL line breaks if available
\IfFileExists{bookmark.sty}{\usepackage{bookmark}}{\usepackage{hyperref}}
\hypersetup{
  pdftitle={Arbeidskrav5},
  pdfauthor={Jacob Mollatt},
  hidelinks,
  pdfcreator={LaTeX via pandoc}}
\urlstyle{same} % disable monospaced font for URLs
\usepackage[margin=1in]{geometry}
\usepackage{graphicx}
\makeatletter
\def\maxwidth{\ifdim\Gin@nat@width>\linewidth\linewidth\else\Gin@nat@width\fi}
\def\maxheight{\ifdim\Gin@nat@height>\textheight\textheight\else\Gin@nat@height\fi}
\makeatother
% Scale images if necessary, so that they will not overflow the page
% margins by default, and it is still possible to overwrite the defaults
% using explicit options in \includegraphics[width, height, ...]{}
\setkeys{Gin}{width=\maxwidth,height=\maxheight,keepaspectratio}
% Set default figure placement to htbp
\makeatletter
\def\fps@figure{htbp}
\makeatother
\setlength{\emergencystretch}{3em} % prevent overfull lines
\providecommand{\tightlist}{%
  \setlength{\itemsep}{0pt}\setlength{\parskip}{0pt}}
\setcounter{secnumdepth}{-\maxdimen} % remove section numbering
\usepackage{booktabs}
\usepackage{longtable}
\usepackage{array}
\usepackage{multirow}
\usepackage{wrapfig}
\usepackage{float}
\usepackage{colortbl}
\usepackage{pdflscape}
\usepackage{tabu}
\usepackage{threeparttable}
\usepackage{threeparttablex}
\usepackage[normalem]{ulem}
\usepackage{makecell}
\usepackage{xcolor}
\usepackage{fontspec}
\usepackage{multicol}
\usepackage{hhline}
\usepackage{hyperref}
\ifluatex
  \usepackage{selnolig}  % disable illegal ligatures
\fi

\title{Arbeidskrav5}
\author{Jacob Mollatt}
\date{11/21/2021}

\begin{document}
\maketitle

\begin{verbatim}
## -- Attaching packages --------------------------------------- tidyverse 1.3.1 --
\end{verbatim}

\begin{verbatim}
## v ggplot2 3.3.5     v purrr   0.3.4
## v tibble  3.1.4     v dplyr   1.0.7
## v tidyr   1.1.3     v stringr 1.4.0
## v readr   2.0.1     v forcats 0.5.1
\end{verbatim}

\begin{verbatim}
## -- Conflicts ------------------------------------------ tidyverse_conflicts() --
## x dplyr::filter() masks stats::filter()
## x dplyr::lag()    masks stats::lag()
\end{verbatim}

\begin{verbatim}
## # A tibble: 68 x 10
##    participant sex    include sets     leg     pre  post lbmleg.change  pre.mc
##    <chr>       <chr>  <chr>   <chr>    <chr> <dbl> <dbl>         <dbl>   <dbl>
##  1 FP28        female incl    multiple L      7059  7273           214 -1537. 
##  2 FP28        female incl    single   R      7104  7227           123 -1492. 
##  3 FP40        female incl    single   L      7190  7192             2 -1406. 
##  4 FP40        female incl    multiple R      7506  7437           -69 -1090. 
##  5 FP21        male   incl    single   L     10281 10470           189  1685. 
##  6 FP21        male   incl    multiple R     10200 10819           619  1604. 
##  7 FP34        female incl    single   L      6014  6326           312 -2582. 
##  8 FP34        female incl    multiple R      6009  6405           396 -2587. 
##  9 FP23        male   incl    single   L      8242  8687           445  -354. 
## 10 FP23        male   incl    multiple R      8685  8480          -205    88.7
## # ... with 58 more rows, and 1 more variable: pros.change <dbl>
\end{verbatim}

\begin{verbatim}
## # A tibble: 2 x 2
##   sets     meansd    
##   <chr>    <chr>     
## 1 multiple 3.37(4.59)
## 2 single   2.05(3.62)
\end{verbatim}

\begin{verbatim}
## [1] "multiple" "single"
\end{verbatim}

\begin{verbatim}
## [1] "3.37(4.59)" "2.05(3.62)"
\end{verbatim}

\begin{verbatim}
## Loading required package: lme4
\end{verbatim}

\begin{verbatim}
## Loading required package: Matrix
\end{verbatim}

\begin{verbatim}
## 
## Attaching package: 'Matrix'
\end{verbatim}

\begin{verbatim}
## The following objects are masked from 'package:tidyr':
## 
##     expand, pack, unpack
\end{verbatim}

\begin{verbatim}
## 
## Attaching package: 'lmerTest'
\end{verbatim}

\begin{verbatim}
## The following object is masked from 'package:lme4':
## 
##     lmer
\end{verbatim}

\begin{verbatim}
## The following object is masked from 'package:stats':
## 
##     step
\end{verbatim}

\begin{verbatim}
## Warning: Some predictor variables are on very different scales: consider
## rescaling
\end{verbatim}

\begin{verbatim}
## Linear mixed model fit by REML ['lmerMod']
## Formula: post ~ pre + sex + sets + (1 | participant)
##    Data: dat
## 
## REML criterion at convergence: 951.6
## 
## Scaled residuals: 
##      Min       1Q   Median       3Q      Max 
## -2.16339 -0.55999  0.03483  0.45253  1.49108 
## 
## Random effects:
##  Groups      Name        Variance Std.Dev.
##  participant (Intercept) 92349    303.9   
##  Residual                51744    227.5   
## Number of obs: 68, groups:  participant, 34
## 
## Fixed effects:
##               Estimate Std. Error t value
## (Intercept)  604.58946  368.30907   1.642
## pre            0.94743    0.05072  18.679
## sexmale      290.59159  203.81270   1.426
## setssingle  -123.55792   55.17543  -2.239
## 
## Correlation of Fixed Effects:
##            (Intr) pre    sexmal
## pre        -0.973              
## sexmale     0.705 -0.815       
## setssingle -0.088  0.013 -0.011
## fit warnings:
## Some predictor variables are on very different scales: consider rescaling
\end{verbatim}

\includegraphics{mixedmethods_files/figure-latex/unnamed-chunk-2-1.pdf}

\begin{verbatim}
## Computing profile confidence intervals ...
\end{verbatim}

\begin{verbatim}
##                   2.5 %      97.5 %
## .sig01       201.703720  396.941748
## .sigma       179.355748  290.662544
## (Intercept) -107.769056 1337.570567
## pre            0.846308    1.045576
## sexmale     -102.219992  693.375809
## setssingle  -233.172861  -13.806501
\end{verbatim}

\begin{verbatim}
## `summarise()` has grouped output by 'participant', 'time', 'sex'. You can override using the `.groups` argument.
\end{verbatim}

\begin{verbatim}
## # A tibble: 78 x 10
##    participant sex    sets      post   pre session1 week2 week5 week9 pros.change
##    <chr>       <chr>  <chr>    <dbl> <dbl>    <dbl> <dbl> <dbl> <dbl>       <dbl>
##  1 FP1         male   multiple 0.696 0.560    0.541 0.572 0.626 0.715        24.2
##  2 FP1         male   single   0.687 0.603    0.628 0.674 0.693 0.722        14.0
##  3 FP11        male   multiple 0.776 0.604    0.594 0.711 0.772 0.737        28.3
##  4 FP11        male   single   0.708 0.568    0.570 0.637 0.693 0.644        24.6
##  5 FP12        female multiple 0.757 0.601    0.627 0.652 0.637 0.715        25.9
##  6 FP12        female single   0.729 0.559    0.600 0.634 0.597 0.680        30.5
##  7 FP13        male   multiple 0.732 0.512    0.528 0.600 0.660 0.698        42.9
##  8 FP13        male   single   0.757 0.531    0.541 0.597 0.673 0.711        42.6
##  9 FP14        female multiple 0.518 0.364    0.324 0.440 0.448 0.511        42.5
## 10 FP14        female single   0.490 0.395    0.382 0.431 0.445 0.470        24.0
## # ... with 68 more rows
\end{verbatim}

\begin{verbatim}
## # A tibble: 2 x 2
##   sets     meansd    
##   <chr>    <chr>     
## 1 multiple 31(14.2)  
## 2 single   24.5(12.9)
\end{verbatim}

\begin{verbatim}
## Linear mixed model fit by REML ['lmerMod']
## Formula: post ~ pre + sex + sets + (1 | participant)
##    Data: strength
## 
## REML criterion at convergence: -221.2
## 
## Scaled residuals: 
##     Min      1Q  Median      3Q     Max 
## -1.8781 -0.4709 -0.1018  0.5958  1.4745 
## 
## Random effects:
##  Groups      Name        Variance  Std.Dev.
##  participant (Intercept) 0.0024439 0.04944 
##  Residual                0.0007301 0.02702 
## Number of obs: 72, groups:  participant, 36
## 
## Fixed effects:
##              Estimate Std. Error t value
## (Intercept)  0.183280   0.038364   4.777
## pre          0.848904   0.091541   9.274
## sexmale      0.067136   0.024689   2.719
## setssingle  -0.029263   0.006378  -4.588
## 
## Correlation of Fixed Effects:
##            (Intr) pre    sexmal
## pre        -0.945              
## sexmale     0.503 -0.697       
## setssingle -0.133  0.053 -0.037
\end{verbatim}

\begin{verbatim}
## Computing profile confidence intervals ...
\end{verbatim}

\begin{verbatim}
##                   2.5 %      97.5 %
## .sig01       0.03594193  0.06298848
## .sigma       0.02136358  0.03426735
## (Intercept)  0.10782107  0.26013926
## pre          0.66502431  1.02963482
## sexmale      0.01927314  0.11613940
## setssingle  -0.04186768 -0.01657808
\end{verbatim}

\includegraphics{mixedmethods_files/figure-latex/unnamed-chunk-3-1.pdf}

\hypertarget{introduksjon}{%
\subsection{Introduksjon}\label{introduksjon}}

Styrketrening er relevant i flere ulike fagmiljøer, og engasjerer mange.
Det kan brukes i rehabilitering, prestasjonsutvikling og folkehelse
{[}@hickey2020,@cirer-sastre2017,@lovell2017,@suetta2007{]}. Det finnes
flere faktorer som kan bidra til å skape en overbelastning av det
nevromuskulære systemet, som igjen vil kunne bidra til
styrkeadaptasjon{[}@rhea2002, @marshall2011{]}. Deriblant skiller kjente
faktorer som volum, intensitet og frekvens seg ut som noen av de
viktigste. Det har blitt gjort flere undersøkelser på hvordan disse
faktorene påvirker eksempelvis maksimal styrke, hypertrofi, og
antropometri, blant disse finner vi også studier som undersøke om man så
en forskjell i hypertrofi gjennom å trene med ett sett kontra tre
sett{[}@schoenfeld2019,@rhea2002; @munn2005;
@fröhlich2010,@carpinelli1998{]}. På tross av en tendens i flere av
disse studiene som viser til at det er observert statisk signifikante
forskjeller i trening av ett sett kontra tre sett, er det også studier
som viser til at det ikke er noen signifikant forskjell. Dette gir oss
et grunnlag for å undersøke om det er forskjeller i treningsresponsen
dersom man gjør styrketrening med ett sett kontra tre sett. Rent
spesifikt ønsker vi å se på om det er forskjeller i styrke og fettfri
masse dersom man trener styrketrening på beina med ett sett kontra tre
sett.

\hypertarget{metode}{%
\subsection{Metode}\label{metode}}

\hypertarget{deltakere}{%
\paragraph{Deltakere}\label{deltakere}}

Det ble rekruttert 34 deltakere til denne studien, og inkludert i
analysen. Disse var ikke-røykende personer mellom 18 og 40 år, hvorav 17
av deltakerne var kvinner. Personer med intoleranse for lokalbedøvelse
eller nedsatt muskelsktyrke som følge av tidligere eller nåværende
skader ble ekskludert. Videre ble personer med treningsfrekvens høyere
enn 1 treningsøkt per uke siste 12 måneder, eller medikamentbruk som kan
påvirke treningsadaptasjon også ekskludert fra å delta. Det ble utelatt
7 personer fra dataanalysen som følge av manglende gjennomføring av
treningsprotokollen. Ved oppstartstesting var det ingen forskjeller i
maks styrke, relatert til kroppsvekt og kroppssammensetning mellom
deltakerne som ble inkludert og deltakerne som ble ekskludert. Alle
deltakerne som ble inkludert rapporterte å ha deltatt i ulike idretter
fra tidligere, og litt over halvparten(n=20) medga at de gjennomførte
regelmessig fysisk aktivitet ved oppstart. Halpvarten av disse opplyste
at denne fysiske aktiviteten var sporadisk styrketrening, men alle under
eksklusjonskriteriet om treningsfrekvens på mer enn 1 treningsøkt per
uke.

\hypertarget{intervensjon-og-testpunkter}{%
\paragraph{Intervensjon og
testpunkter}\label{intervensjon-og-testpunkter}}

Deltakerne fulgte en treningsintervensjon over 12 uker. Etter oppstart
ble styrketesting gjennomført etter henhodlsvis 3, 5 og 9 uker. Det ble
også gjennomført styrke etter treningsintervensjonen var avsluttet. I
tillegg til styrketesting ble deltakernes kropssammensetning mål før og
etter treningsintervensjonen, i tillegg til at det ble hentet
muskelbiopsier av vastus lateralis fra alle deltakere. Muskelbiopsier
ble gjennomført på følgende tidspunkter; ved oppstart(Uke 0), før femte
treningsøkt(Uke 2), 1 time etter femte treningsøkt(Uke 2), og etter
intervensjonen var avsluttet(Uke 12).

Styrkeøvelsene ble gjennomført unilateralt for at pasientene kunne
gjennomføre ulike protokoller på ulike bein. I forkant av alle
treningsøkter gjennomførte deltakerne en standardisert og progressiv
oppvarmingsprotokoll. Deltakerne gjennomførte øvelser i følgende
rekkefølge; unilateral beinpress, knefleksjon og kneekstensjon.
Deltakerne gjorde ett sett på ett bein, og tre sett på det motsatte
beinet. Enkeltsettet ble gjennomført mellom andre og tredje sett på
motsatt bein. Underveis i intervensjonen ble intensiteten økt gradvis
fra 10RM til 8RM og deretter til 7 RM.

\hypertarget{muxe5linger}{%
\paragraph{Målinger}\label{muxe5linger}}

Det ble gjennomført isokinetiske og isometriske styrketester av styrken
i kneekstensjon i dynamometer (Cybex). Maks styrke ble mål som 1RM i
unilateralt benpress og kneektesjon. Muskeltversnitt i
kneekstensorgruppen ble målt før og etter treningsintervensjon gjennom
MR. Kroppssammensetning ble målt ved hjelp av DXA. Muskelbiopsi ble
hentet bilateralt fra vastus lateralis.

\hypertarget{statistiske-analyser}{%
\paragraph{Statistiske analyser}\label{statistiske-analyser}}

For å sammenligne hvilken effekt styrketrening med kun ett enkeltsett
kontra flere sett har på muskelstyrken i beina, og mengden
muskelmasse(fettfri masse) har vi i denne studien sett på prosentvis
endring fra baseline til etter intervensjon. Det har blitt regnet ut og
presentert den gjennomsnittlige endringen fra baseline til etter
intervensjon, med standarddavvik. De ulike intervensjonene er også
sammenlignet med hjelp av en ANVOCA-modell, for å ta høyde for
samvarierende variabler. Regresjonsligningen skal se på endringen etter
at man tar høyde for forskjell i verdier ved pre-test og etter å ha tatt
høyde for forskjell i kjønn. Det har også bitt foretatt en
gjenomsnittssentrering av dataene.

\begin{verbatim}
## 
## Attaching package: 'kableExtra'
\end{verbatim}

\begin{verbatim}
## The following object is masked from 'package:dplyr':
## 
##     group_rows
\end{verbatim}

\begin{verbatim}
## 
## Attaching package: 'flextable'
\end{verbatim}

\begin{verbatim}
## The following objects are masked from 'package:kableExtra':
## 
##     as_image, footnote
\end{verbatim}

\begin{verbatim}
## The following object is masked from 'package:purrr':
## 
##     compose
\end{verbatim}

\providecommand{\docline}[3]{\noalign{\global\setlength{\arrayrulewidth}{#1}}\arrayrulecolor[HTML]{#2}\cline{#3}}

\setlength{\tabcolsep}{2pt}

\renewcommand*{\arraystretch}{1.5}

\begin{longtable}[c]{|p{0.96in}|p{2.95in}}



\hhline{>{\arrayrulecolor[HTML]{666666}\global\arrayrulewidth=2pt}->{\arrayrulecolor[HTML]{666666}\global\arrayrulewidth=2pt}-}

\multicolumn{1}{!{\color[HTML]{000000}\vrule width 0pt}>{\raggedright}p{\dimexpr 0.96in+0\tabcolsep+0\arrayrulewidth}}{\fontsize{11}{11}\selectfont{\textcolor[HTML]{000000}{\global\setmainfont{Helvetica}{\ }}}} & \multicolumn{1}{!{\color[HTML]{000000}\vrule width 0pt}>{\raggedright}p{\dimexpr 2.95in+0\tabcolsep+0\arrayrulewidth}!{\color[HTML]{000000}\vrule width 0pt}}{\fontsize{11}{11}\selectfont{\textcolor[HTML]{000000}{\global\setmainfont{Helvetica}{Tabell\ 2\ -\ Økning\ fettfri\ masse\ i\ prosent}}}} \\

\noalign{\global\setlength{\arrayrulewidth}{2pt}}\arrayrulecolor[HTML]{666666}\cline{1-2}

\endfirsthead

\hhline{>{\arrayrulecolor[HTML]{666666}\global\arrayrulewidth=2pt}->{\arrayrulecolor[HTML]{666666}\global\arrayrulewidth=2pt}-}

\multicolumn{1}{!{\color[HTML]{000000}\vrule width 0pt}>{\raggedright}p{\dimexpr 0.96in+0\tabcolsep+0\arrayrulewidth}}{\fontsize{11}{11}\selectfont{\textcolor[HTML]{000000}{\global\setmainfont{Helvetica}{\ }}}} & \multicolumn{1}{!{\color[HTML]{000000}\vrule width 0pt}>{\raggedright}p{\dimexpr 2.95in+0\tabcolsep+0\arrayrulewidth}!{\color[HTML]{000000}\vrule width 0pt}}{\fontsize{11}{11}\selectfont{\textcolor[HTML]{000000}{\global\setmainfont{Helvetica}{Tabell\ 2\ -\ Økning\ fettfri\ masse\ i\ prosent}}}} \\

\noalign{\global\setlength{\arrayrulewidth}{2pt}}\arrayrulecolor[HTML]{666666}\cline{1-2}\endhead



\multicolumn{2}{!{\color[HTML]{FFFFFF}\vrule width 0pt}>{\raggedright}p{\dimexpr 3.91in+2\tabcolsep+1\arrayrulewidth}!{\color[HTML]{FFFFFF}\vrule width 0pt}}{\fontsize{11}{11}\selectfont{\textcolor[HTML]{000000}{\global\setmainfont{Helvetica}{Gjennomsnittlig\ prosentvis\ endring(SD),\ fra\ pre\ til\ post}}}} \\

\endfoot



\multicolumn{1}{!{\color[HTML]{000000}\vrule width 0pt}>{\raggedright}p{\dimexpr 0.96in+0\tabcolsep+0\arrayrulewidth}}{\fontsize{11}{11}\selectfont{\textcolor[HTML]{000000}{\global\setmainfont{Helvetica}{Flersett}}}} & \multicolumn{1}{!{\color[HTML]{000000}\vrule width 0pt}>{\raggedright}p{\dimexpr 2.95in+0\tabcolsep+0\arrayrulewidth}!{\color[HTML]{000000}\vrule width 0pt}}{\fontsize{11}{11}\selectfont{\textcolor[HTML]{000000}{\global\setmainfont{Helvetica}{3.37(4.59)}}}} \\





\multicolumn{1}{!{\color[HTML]{000000}\vrule width 0pt}>{\raggedright}p{\dimexpr 0.96in+0\tabcolsep+0\arrayrulewidth}}{\fontsize{11}{11}\selectfont{\textcolor[HTML]{000000}{\global\setmainfont{Helvetica}{Enkeltsett}}}} & \multicolumn{1}{!{\color[HTML]{000000}\vrule width 0pt}>{\raggedright}p{\dimexpr 2.95in+0\tabcolsep+0\arrayrulewidth}!{\color[HTML]{000000}\vrule width 0pt}}{\fontsize{11}{11}\selectfont{\textcolor[HTML]{000000}{\global\setmainfont{Helvetica}{2.05(3.62)}}}} \\

\noalign{\global\setlength{\arrayrulewidth}{2pt}}\arrayrulecolor[HTML]{666666}\cline{1-2}



\end{longtable}

\providecommand{\docline}[3]{\noalign{\global\setlength{\arrayrulewidth}{#1}}\arrayrulecolor[HTML]{#2}\cline{#3}}

\setlength{\tabcolsep}{2pt}

\renewcommand*{\arraystretch}{1.5}

\begin{longtable}[c]{|p{0.96in}|p{2.82in}}



\hhline{>{\arrayrulecolor[HTML]{666666}\global\arrayrulewidth=2pt}->{\arrayrulecolor[HTML]{666666}\global\arrayrulewidth=2pt}-}

\multicolumn{1}{!{\color[HTML]{000000}\vrule width 0pt}>{\raggedright}p{\dimexpr 0.96in+0\tabcolsep+0\arrayrulewidth}}{\fontsize{11}{11}\selectfont{\textcolor[HTML]{000000}{\global\setmainfont{Helvetica}{\ }}}} & \multicolumn{1}{!{\color[HTML]{000000}\vrule width 0pt}>{\raggedright}p{\dimexpr 2.82in+0\tabcolsep+0\arrayrulewidth}!{\color[HTML]{000000}\vrule width 0pt}}{\fontsize{11}{11}\selectfont{\textcolor[HTML]{000000}{\global\setmainfont{Helvetica}{Tabell\ 3\ -\ Økning\ beinstyrke\ i\ prosent}}}} \\

\noalign{\global\setlength{\arrayrulewidth}{2pt}}\arrayrulecolor[HTML]{666666}\cline{1-2}

\endfirsthead

\hhline{>{\arrayrulecolor[HTML]{666666}\global\arrayrulewidth=2pt}->{\arrayrulecolor[HTML]{666666}\global\arrayrulewidth=2pt}-}

\multicolumn{1}{!{\color[HTML]{000000}\vrule width 0pt}>{\raggedright}p{\dimexpr 0.96in+0\tabcolsep+0\arrayrulewidth}}{\fontsize{11}{11}\selectfont{\textcolor[HTML]{000000}{\global\setmainfont{Helvetica}{\ }}}} & \multicolumn{1}{!{\color[HTML]{000000}\vrule width 0pt}>{\raggedright}p{\dimexpr 2.82in+0\tabcolsep+0\arrayrulewidth}!{\color[HTML]{000000}\vrule width 0pt}}{\fontsize{11}{11}\selectfont{\textcolor[HTML]{000000}{\global\setmainfont{Helvetica}{Tabell\ 3\ -\ Økning\ beinstyrke\ i\ prosent}}}} \\

\noalign{\global\setlength{\arrayrulewidth}{2pt}}\arrayrulecolor[HTML]{666666}\cline{1-2}\endhead



\multicolumn{2}{!{\color[HTML]{FFFFFF}\vrule width 0pt}>{\raggedright}p{\dimexpr 3.77in+2\tabcolsep+1\arrayrulewidth}!{\color[HTML]{FFFFFF}\vrule width 0pt}}{\fontsize{11}{11}\selectfont{\textcolor[HTML]{000000}{\global\setmainfont{Helvetica}{Gjennomsnittlig\ prosentvis\ endring(SD),\ fra\ pre\ til\ post}}}} \\

\endfoot



\multicolumn{1}{!{\color[HTML]{000000}\vrule width 0pt}>{\raggedright}p{\dimexpr 0.96in+0\tabcolsep+0\arrayrulewidth}}{\fontsize{11}{11}\selectfont{\textcolor[HTML]{000000}{\global\setmainfont{Helvetica}{Flersett}}}} & \multicolumn{1}{!{\color[HTML]{000000}\vrule width 0pt}>{\raggedright}p{\dimexpr 2.82in+0\tabcolsep+0\arrayrulewidth}!{\color[HTML]{000000}\vrule width 0pt}}{\fontsize{11}{11}\selectfont{\textcolor[HTML]{000000}{\global\setmainfont{Helvetica}{31(14.2)}}}} \\





\multicolumn{1}{!{\color[HTML]{000000}\vrule width 0pt}>{\raggedright}p{\dimexpr 0.96in+0\tabcolsep+0\arrayrulewidth}}{\fontsize{11}{11}\selectfont{\textcolor[HTML]{000000}{\global\setmainfont{Helvetica}{Enkeltsett}}}} & \multicolumn{1}{!{\color[HTML]{000000}\vrule width 0pt}>{\raggedright}p{\dimexpr 2.82in+0\tabcolsep+0\arrayrulewidth}!{\color[HTML]{000000}\vrule width 0pt}}{\fontsize{11}{11}\selectfont{\textcolor[HTML]{000000}{\global\setmainfont{Helvetica}{24.5(12.9)}}}} \\

\noalign{\global\setlength{\arrayrulewidth}{2pt}}\arrayrulecolor[HTML]{666666}\cline{1-2}



\end{longtable}

\hypertarget{resultater}{%
\subsection{Resultater}\label{resultater}}

\hypertarget{muskelmasse}{%
\paragraph{Muskelmasse}\label{muskelmasse}}

Etter å ha justert for kjønn og baselineverdier ser man at det beinet
deltakerne brukte til å trene flere sett gjennomsnitt økt muskelmassen
med 123,5 gram mer(SD 55.17 gram, 95\% CI), enn beinet som gjennomførte
kun ett sett. Gjennomsnittlig prosentvis endring i fra baselinetesting
til fullført intervensjon er vist i Tabell 2

\providecommand{\docline}[3]{\noalign{\global\setlength{\arrayrulewidth}{#1}}\arrayrulecolor[HTML]{#2}\cline{#3}}

\setlength{\tabcolsep}{2pt}

\renewcommand*{\arraystretch}{1.5}

\begin{longtable}[c]{|p{0.96in}|p{2.95in}}



\hhline{>{\arrayrulecolor[HTML]{666666}\global\arrayrulewidth=2pt}->{\arrayrulecolor[HTML]{666666}\global\arrayrulewidth=2pt}-}

\multicolumn{1}{!{\color[HTML]{000000}\vrule width 0pt}>{\raggedright}p{\dimexpr 0.96in+0\tabcolsep+0\arrayrulewidth}}{\fontsize{11}{11}\selectfont{\textcolor[HTML]{000000}{\global\setmainfont{Helvetica}{\ }}}} & \multicolumn{1}{!{\color[HTML]{000000}\vrule width 0pt}>{\raggedright}p{\dimexpr 2.95in+0\tabcolsep+0\arrayrulewidth}!{\color[HTML]{000000}\vrule width 0pt}}{\fontsize{11}{11}\selectfont{\textcolor[HTML]{000000}{\global\setmainfont{Helvetica}{Tabell\ 2\ -\ Økning\ fettfri\ masse\ i\ prosent}}}} \\

\noalign{\global\setlength{\arrayrulewidth}{2pt}}\arrayrulecolor[HTML]{666666}\cline{1-2}

\endfirsthead

\hhline{>{\arrayrulecolor[HTML]{666666}\global\arrayrulewidth=2pt}->{\arrayrulecolor[HTML]{666666}\global\arrayrulewidth=2pt}-}

\multicolumn{1}{!{\color[HTML]{000000}\vrule width 0pt}>{\raggedright}p{\dimexpr 0.96in+0\tabcolsep+0\arrayrulewidth}}{\fontsize{11}{11}\selectfont{\textcolor[HTML]{000000}{\global\setmainfont{Helvetica}{\ }}}} & \multicolumn{1}{!{\color[HTML]{000000}\vrule width 0pt}>{\raggedright}p{\dimexpr 2.95in+0\tabcolsep+0\arrayrulewidth}!{\color[HTML]{000000}\vrule width 0pt}}{\fontsize{11}{11}\selectfont{\textcolor[HTML]{000000}{\global\setmainfont{Helvetica}{Tabell\ 2\ -\ Økning\ fettfri\ masse\ i\ prosent}}}} \\

\noalign{\global\setlength{\arrayrulewidth}{2pt}}\arrayrulecolor[HTML]{666666}\cline{1-2}\endhead



\multicolumn{2}{!{\color[HTML]{FFFFFF}\vrule width 0pt}>{\raggedright}p{\dimexpr 3.91in+2\tabcolsep+1\arrayrulewidth}!{\color[HTML]{FFFFFF}\vrule width 0pt}}{\fontsize{11}{11}\selectfont{\textcolor[HTML]{000000}{\global\setmainfont{Helvetica}{Gjennomsnittlig\ prosentvis\ endring(SD),\ fra\ pre\ til\ post}}}} \\

\endfoot



\multicolumn{1}{!{\color[HTML]{000000}\vrule width 0pt}>{\raggedright}p{\dimexpr 0.96in+0\tabcolsep+0\arrayrulewidth}}{\fontsize{11}{11}\selectfont{\textcolor[HTML]{000000}{\global\setmainfont{Helvetica}{Flersett}}}} & \multicolumn{1}{!{\color[HTML]{000000}\vrule width 0pt}>{\raggedright}p{\dimexpr 2.95in+0\tabcolsep+0\arrayrulewidth}!{\color[HTML]{000000}\vrule width 0pt}}{\fontsize{11}{11}\selectfont{\textcolor[HTML]{000000}{\global\setmainfont{Helvetica}{3.37(4.59)}}}} \\





\multicolumn{1}{!{\color[HTML]{000000}\vrule width 0pt}>{\raggedright}p{\dimexpr 0.96in+0\tabcolsep+0\arrayrulewidth}}{\fontsize{11}{11}\selectfont{\textcolor[HTML]{000000}{\global\setmainfont{Helvetica}{Enkeltsett}}}} & \multicolumn{1}{!{\color[HTML]{000000}\vrule width 0pt}>{\raggedright}p{\dimexpr 2.95in+0\tabcolsep+0\arrayrulewidth}!{\color[HTML]{000000}\vrule width 0pt}}{\fontsize{11}{11}\selectfont{\textcolor[HTML]{000000}{\global\setmainfont{Helvetica}{2.05(3.62)}}}} \\

\noalign{\global\setlength{\arrayrulewidth}{2pt}}\arrayrulecolor[HTML]{666666}\cline{1-2}



\end{longtable}

.

\includegraphics{mixedmethods_files/figure-latex/unnamed-chunk-5-1.pdf}

\hypertarget{styrke}{%
\paragraph{Styrke}\label{styrke}}

Etter å ha justert for verdier ved baseline og kjønn så man at beinet
som har gjennomført trening med enkeltsett hadde en signifikant lavere
styrke etter avsluttet intervensjon enn beinet som hadde gjennomført
trening med flersett. Fremgangen er målt i skalert motstand, som er
motstand som en andel av maks. I skalert motstand hadde beinet med
enkeltsett gjennomsnittlig 0.029(95\% CI) lavere verdier etter endt
intervensjon enn beinet med flersett.

Gjennomsnittlig prosentvis endring i muskelstyrke fra baselinetesting
til fullført intervensjon er vist i Tabell 3

\providecommand{\docline}[3]{\noalign{\global\setlength{\arrayrulewidth}{#1}}\arrayrulecolor[HTML]{#2}\cline{#3}}

\setlength{\tabcolsep}{2pt}

\renewcommand*{\arraystretch}{1.5}

\begin{longtable}[c]{|p{0.96in}|p{2.82in}}



\hhline{>{\arrayrulecolor[HTML]{666666}\global\arrayrulewidth=2pt}->{\arrayrulecolor[HTML]{666666}\global\arrayrulewidth=2pt}-}

\multicolumn{1}{!{\color[HTML]{000000}\vrule width 0pt}>{\raggedright}p{\dimexpr 0.96in+0\tabcolsep+0\arrayrulewidth}}{\fontsize{11}{11}\selectfont{\textcolor[HTML]{000000}{\global\setmainfont{Helvetica}{\ }}}} & \multicolumn{1}{!{\color[HTML]{000000}\vrule width 0pt}>{\raggedright}p{\dimexpr 2.82in+0\tabcolsep+0\arrayrulewidth}!{\color[HTML]{000000}\vrule width 0pt}}{\fontsize{11}{11}\selectfont{\textcolor[HTML]{000000}{\global\setmainfont{Helvetica}{Tabell\ 3\ -\ Økning\ beinstyrke\ i\ prosent}}}} \\

\noalign{\global\setlength{\arrayrulewidth}{2pt}}\arrayrulecolor[HTML]{666666}\cline{1-2}

\endfirsthead

\hhline{>{\arrayrulecolor[HTML]{666666}\global\arrayrulewidth=2pt}->{\arrayrulecolor[HTML]{666666}\global\arrayrulewidth=2pt}-}

\multicolumn{1}{!{\color[HTML]{000000}\vrule width 0pt}>{\raggedright}p{\dimexpr 0.96in+0\tabcolsep+0\arrayrulewidth}}{\fontsize{11}{11}\selectfont{\textcolor[HTML]{000000}{\global\setmainfont{Helvetica}{\ }}}} & \multicolumn{1}{!{\color[HTML]{000000}\vrule width 0pt}>{\raggedright}p{\dimexpr 2.82in+0\tabcolsep+0\arrayrulewidth}!{\color[HTML]{000000}\vrule width 0pt}}{\fontsize{11}{11}\selectfont{\textcolor[HTML]{000000}{\global\setmainfont{Helvetica}{Tabell\ 3\ -\ Økning\ beinstyrke\ i\ prosent}}}} \\

\noalign{\global\setlength{\arrayrulewidth}{2pt}}\arrayrulecolor[HTML]{666666}\cline{1-2}\endhead



\multicolumn{2}{!{\color[HTML]{FFFFFF}\vrule width 0pt}>{\raggedright}p{\dimexpr 3.77in+2\tabcolsep+1\arrayrulewidth}!{\color[HTML]{FFFFFF}\vrule width 0pt}}{\fontsize{11}{11}\selectfont{\textcolor[HTML]{000000}{\global\setmainfont{Helvetica}{Gjennomsnittlig\ prosentvis\ endring(SD),\ fra\ pre\ til\ post}}}} \\

\endfoot



\multicolumn{1}{!{\color[HTML]{000000}\vrule width 0pt}>{\raggedright}p{\dimexpr 0.96in+0\tabcolsep+0\arrayrulewidth}}{\fontsize{11}{11}\selectfont{\textcolor[HTML]{000000}{\global\setmainfont{Helvetica}{Flersett}}}} & \multicolumn{1}{!{\color[HTML]{000000}\vrule width 0pt}>{\raggedright}p{\dimexpr 2.82in+0\tabcolsep+0\arrayrulewidth}!{\color[HTML]{000000}\vrule width 0pt}}{\fontsize{11}{11}\selectfont{\textcolor[HTML]{000000}{\global\setmainfont{Helvetica}{31(14.2)}}}} \\





\multicolumn{1}{!{\color[HTML]{000000}\vrule width 0pt}>{\raggedright}p{\dimexpr 0.96in+0\tabcolsep+0\arrayrulewidth}}{\fontsize{11}{11}\selectfont{\textcolor[HTML]{000000}{\global\setmainfont{Helvetica}{Enkeltsett}}}} & \multicolumn{1}{!{\color[HTML]{000000}\vrule width 0pt}>{\raggedright}p{\dimexpr 2.82in+0\tabcolsep+0\arrayrulewidth}!{\color[HTML]{000000}\vrule width 0pt}}{\fontsize{11}{11}\selectfont{\textcolor[HTML]{000000}{\global\setmainfont{Helvetica}{24.5(12.9)}}}} \\

\noalign{\global\setlength{\arrayrulewidth}{2pt}}\arrayrulecolor[HTML]{666666}\cline{1-2}



\end{longtable}

.

\hypertarget{konklusjon}{%
\subsection{Konklusjon}\label{konklusjon}}

Fra baseline til etter intervensjonen var avsluttet var det en
signifikant forskjell i fremgangen i beinet som gjennomførte trening med
flersett sammenlignet med beinet som gjennomførte trening med
enkeltsett. Dette gjaldt både mengden muskelmasse og muskelstyrke.

\hypertarget{kommentar}{%
\paragraph{Kommentar}\label{kommentar}}

Leverer heller et uferdig dokument til fristen, fullstendig klar over
store mangler med referering og fullverdighet i introduksjon og
diskusjon. Håper på tilbakemeldinger på tolkning av resultater og valg
av statistiske tester.

\hypertarget{kommentar-fra-daniel}{%
\subparagraph{KOMMENTAR FRA DANIEL}\label{kommentar-fra-daniel}}

En bra start! Mange ev code-chunksen inngår som det er nå i rapporten,
sett knitr::opts\_chunk\$set(echo = FALSE) for å ha koden i bakgrunnen.

Unngå å starte setninger med et tall, evt skriv ut tallet. Husk
mellomrom før parenteser. Du trenger ikke å rapportere metode på data
som du ikke presenterer i rapporten (eks ernæringsdata).

Unngå å benevne de ulike treningsprotokollene som to ulike grupper.

De samme deltakerne gjennomførte begge protokoller. Hva betyr: 123,5
gram mer(SD 55.17, 95\% CI)? Når du lager resultatene mer fullstendig,
husk å trekke inferens på forskjell mellom treningsprotokoller og bruk
deskriptiv statistikk på resterende data. Se og generell feedback på
github (IDR4000-2021)

\begin{verbatim}
## `summarise()` has grouped output by 'sets'. You can override using the `.groups` argument.
\end{verbatim}

\includegraphics{mixedmethods_files/figure-latex/unnamed-chunk-6-1.pdf}

\end{document}
