% Options for packages loaded elsewhere
\PassOptionsToPackage{unicode}{hyperref}
\PassOptionsToPackage{hyphens}{url}
%
\documentclass[
]{article}
\usepackage{amsmath,amssymb}
\usepackage{lmodern}
\usepackage{ifxetex,ifluatex}
\ifnum 0\ifxetex 1\fi\ifluatex 1\fi=0 % if pdftex
  \usepackage[T1]{fontenc}
  \usepackage[utf8]{inputenc}
  \usepackage{textcomp} % provide euro and other symbols
\else % if luatex or xetex
  \usepackage{unicode-math}
  \defaultfontfeatures{Scale=MatchLowercase}
  \defaultfontfeatures[\rmfamily]{Ligatures=TeX,Scale=1}
\fi
% Use upquote if available, for straight quotes in verbatim environments
\IfFileExists{upquote.sty}{\usepackage{upquote}}{}
\IfFileExists{microtype.sty}{% use microtype if available
  \usepackage[]{microtype}
  \UseMicrotypeSet[protrusion]{basicmath} % disable protrusion for tt fonts
}{}
\makeatletter
\@ifundefined{KOMAClassName}{% if non-KOMA class
  \IfFileExists{parskip.sty}{%
    \usepackage{parskip}
  }{% else
    \setlength{\parindent}{0pt}
    \setlength{\parskip}{6pt plus 2pt minus 1pt}}
}{% if KOMA class
  \KOMAoptions{parskip=half}}
\makeatother
\usepackage{xcolor}
\IfFileExists{xurl.sty}{\usepackage{xurl}}{} % add URL line breaks if available
\IfFileExists{bookmark.sty}{\usepackage{bookmark}}{\usepackage{hyperref}}
\hypersetup{
  pdftitle={Reliabilitet},
  pdfauthor={Jacob Mollatt},
  hidelinks,
  pdfcreator={LaTeX via pandoc}}
\urlstyle{same} % disable monospaced font for URLs
\usepackage[margin=1in]{geometry}
\usepackage{graphicx}
\makeatletter
\def\maxwidth{\ifdim\Gin@nat@width>\linewidth\linewidth\else\Gin@nat@width\fi}
\def\maxheight{\ifdim\Gin@nat@height>\textheight\textheight\else\Gin@nat@height\fi}
\makeatother
% Scale images if necessary, so that they will not overflow the page
% margins by default, and it is still possible to overwrite the defaults
% using explicit options in \includegraphics[width, height, ...]{}
\setkeys{Gin}{width=\maxwidth,height=\maxheight,keepaspectratio}
% Set default figure placement to htbp
\makeatletter
\def\fps@figure{htbp}
\makeatother
\setlength{\emergencystretch}{3em} % prevent overfull lines
\providecommand{\tightlist}{%
  \setlength{\itemsep}{0pt}\setlength{\parskip}{0pt}}
\setcounter{secnumdepth}{-\maxdimen} % remove section numbering
\usepackage{fontspec}
\usepackage{multirow}
\usepackage{multicol}
\usepackage{colortbl}
\usepackage{hhline}
\usepackage{longtable}
\usepackage{array}
\usepackage{hyperref}
\ifluatex
  \usepackage{selnolig}  % disable illegal ligatures
\fi

\title{Reliabilitet}
\author{Jacob Mollatt}
\date{11/29/2021}

\begin{document}
\maketitle

\begin{center}\rule{0.5\linewidth}{0.5pt}\end{center}

\hypertarget{introduksjon}{%
\subsection{Introduksjon}\label{introduksjon}}

Maksimalt oksygenopptak (VO2max) ble først beskrevet av Hill og Lupton i
1923, og kan defineres som kroppens evne til å ta opp og forbruke
oksygen per tidsenhet {[}@bassett2000; @hill1923{]}. Innen toppidrett
måles ofte det maksimale oksygenopptaket for å måle utøverens kapasitet
opp mot arbeidskravet i den spesifikke idretten, og VO2max kan i så måte
også sees på som et mål på den aerobe effekten til utøveren
{[}@bassett2000{]}. I Olympiatoppens testprotokoller benytter de flere
definerte hjelpekriterier for å sikre at man faktisk har funnet
deltakerens maksimale oksygenopptak {[}@tønnessen2017{]}. Følgende
kriterier er beskrevet; platå i O2 er oppnådd, økning i ventilasjon med
utflating av O2 verdi, RER-verdi over 1.10 (1.05 om gjennomført
laktatprofiltest i forkant) og blodlaktat over 8 {[}@tønnessen2017{]}.

\hypertarget{metode}{%
\subsection{Metode}\label{metode}}

I forkant av testen målte alle deltakerne kroppsvekten i samme klær som
ble brukt under testen, men ble bedt om å ta av seg skoene. Kroppsvekten
som senere brukt i beregningen av maksimalt oksygenopptak (ml
kg\textsuperscript{-1} min\textsuperscript{-1}) er kroppsvekten målt i
forkant av test. For å sikre intern validitet ble deltakerne bedt om å
avstå fra anstrengende fysisk aktivitet dagen før test, standardisere
måltidet i forkant av test samt avstå fra inntak av koffein under de
siste 12 timene før testen {[}@halperin2015{]} . Test 1 og Teat 2 ble
gjennomført på samme tid på døgnet under standardiserte forhold.. Test 2
ble gjennomført 6 dager etter gjennomført Test 1. Det ble ikke
kontrollert for fysisk aktivitet mellom testdagene.

Alle deltakerne gjennomførte en 10 minutter lang oppvarmingsprotokoll på
tredemøllen (Woodway 4Front, Waukesha, USA), beskrevet for deltakerne i
forkant av testen. Denne oppvarmingsprotokollen bestod av fem minutter
på 11-13 i Borg 6-20 RPE skala {[}@borg1982{]}, etterfulgt av 2 drag med
varighet på 1 minutt. Disse dragene var på samme hastighet og stigning
som ved teststart, og ble adskilt med 30 sekunders pause hvor deltakerne
sto i ro. De tre siste minuttetene av oppvarmingen var igjen begrenset
til en intensitet mellom 11 og 12 på Borg skala. Etter gjennomført
oppvarmingsprotokoll fikk deltakerne en pause på 2 minutter før testen
begynte. Starthastighet for begge kjønn var satt til 8km/t, med stigning
på 10.5\% og 5.5\% for henholdsvis menn og kvinner.

VO2max ble målt ved hjelp av en metabolsk analysator med miksekammer
(vyntus CPX, mixing chamber (Vyntus CPX, Jaeger-CareFusion, UK). Forut
for alle tester ble analysatoren gass- og volumkalibrert. Analysatoren
ble stilt inn til å gjøre målinger hvert 30 sekunc, og VO2max ble
kalkulert gjennom å bruke snittet av de to høyeste påfølgende målingene
av O2. Det ble tatt en avgjørelse på at alle deltakerne underveis i
testen mottok en høylytt verbal oppmuntring fra testleder, selv om dette
vil ha en påvirkning på resultatene {[}@halperin2015{]}. Dette kan
forsvares gjennom at man ønsket å bidra til at deltakerne faktisk
gjennomførte testen til utmattelse, noe som også kan argumenteres for å
være viktig å få valide resultater {[}@halperin2015,@tønnessen2017{]}.
Dette ble gjort både på Test 1 og Test 2. Alle deltakerne gjennomførte
også begge testene med samme testleder og med samme personer til stede i
rommet for å redusere konfundering {[}@halperin2015{]}.

For hvert medgåtte minutt av testen ble hastigheten på møllen økt med
1km/t, helt til utmattelse, hvor testen ble avsluttet. Deltakernes
hjertefrekvens ble også registrert under hele testen. Når testen ble
avsluttet ble deltakerne bedt om å rapportere opplevd anstrengelse ved
hjelp av Borg-skala {[}@borg1982{]}. Maksimal hjertefrekvens under
testen ble også registrert. Ett minutt etter avsluttet test ble
hjertefrekvens registrert, og det ble målt og analysert blodlaktat
(Biosen C-line, EKF Diagnostics, Barleben, Germany).

Det var 11 deltakere i studien, samtlige deltakere er studenter ved
Høgskolen i Innlandet. Deskriptive data for disse deltakerne er vist i
Tabell 1.

\hypertarget{resultater}{%
\subsection{Resultater}\label{resultater}}

\providecommand{\docline}[3]{\noalign{\global\setlength{\arrayrulewidth}{#1}}\arrayrulecolor[HTML]{#2}\cline{#3}}

\setlength{\tabcolsep}{2pt}

\renewcommand*{\arraystretch}{1.5}

\begin{longtable}[c]{|p{1.08in}|p{1.02in}|p{1.02in}}



\hhline{>{\arrayrulecolor[HTML]{666666}\global\arrayrulewidth=2pt}->{\arrayrulecolor[HTML]{666666}\global\arrayrulewidth=2pt}->{\arrayrulecolor[HTML]{666666}\global\arrayrulewidth=2pt}-}

\multicolumn{1}{!{\color[HTML]{000000}\vrule width 0pt}>{\raggedright}p{\dimexpr 1.08in+0\tabcolsep+0\arrayrulewidth}}{\fontsize{11}{11}\selectfont{\textcolor[HTML]{000000}{\global\setmainfont{Helvetica}{}}}} & \multicolumn{1}{!{\color[HTML]{000000}\vrule width 0pt}>{\raggedright}p{\dimexpr 1.02in+0\tabcolsep+0\arrayrulewidth}}{\fontsize{11}{11}\selectfont{\textcolor[HTML]{000000}{\global\setmainfont{Helvetica}{Kvinner}}}} & \multicolumn{1}{!{\color[HTML]{000000}\vrule width 0pt}>{\raggedright}p{\dimexpr 1.02in+0\tabcolsep+0\arrayrulewidth}!{\color[HTML]{000000}\vrule width 0pt}}{\fontsize{11}{11}\selectfont{\textcolor[HTML]{000000}{\global\setmainfont{Helvetica}{Menn}}}} \\

\noalign{\global\setlength{\arrayrulewidth}{2pt}}\arrayrulecolor[HTML]{666666}\cline{1-3}

\endfirsthead

\hhline{>{\arrayrulecolor[HTML]{666666}\global\arrayrulewidth=2pt}->{\arrayrulecolor[HTML]{666666}\global\arrayrulewidth=2pt}->{\arrayrulecolor[HTML]{666666}\global\arrayrulewidth=2pt}-}

\multicolumn{1}{!{\color[HTML]{000000}\vrule width 0pt}>{\raggedright}p{\dimexpr 1.08in+0\tabcolsep+0\arrayrulewidth}}{\fontsize{11}{11}\selectfont{\textcolor[HTML]{000000}{\global\setmainfont{Helvetica}{}}}} & \multicolumn{1}{!{\color[HTML]{000000}\vrule width 0pt}>{\raggedright}p{\dimexpr 1.02in+0\tabcolsep+0\arrayrulewidth}}{\fontsize{11}{11}\selectfont{\textcolor[HTML]{000000}{\global\setmainfont{Helvetica}{Kvinner}}}} & \multicolumn{1}{!{\color[HTML]{000000}\vrule width 0pt}>{\raggedright}p{\dimexpr 1.02in+0\tabcolsep+0\arrayrulewidth}!{\color[HTML]{000000}\vrule width 0pt}}{\fontsize{11}{11}\selectfont{\textcolor[HTML]{000000}{\global\setmainfont{Helvetica}{Menn}}}} \\

\noalign{\global\setlength{\arrayrulewidth}{2pt}}\arrayrulecolor[HTML]{666666}\cline{1-3}\endhead



\multicolumn{3}{!{\color[HTML]{FFFFFF}\vrule width 0pt}>{\raggedright}p{\dimexpr 3.13in+4\tabcolsep+2\arrayrulewidth}!{\color[HTML]{FFFFFF}\vrule width 0pt}}{\fontsize{11}{11}\selectfont{\textcolor[HTML]{000000}{\global\setmainfont{Helvetica}{Verdier\ er\ gitt\ som\ gjennomsnitt\ og\ (Standardavvik)}}}} \\

\endfoot



\multicolumn{1}{!{\color[HTML]{000000}\vrule width 0pt}>{\raggedright}p{\dimexpr 1.08in+0\tabcolsep+0\arrayrulewidth}}{\fontsize{11}{11}\selectfont{\textcolor[HTML]{000000}{\global\setmainfont{Helvetica}{N}}}} & \multicolumn{1}{!{\color[HTML]{000000}\vrule width 0pt}>{\raggedright}p{\dimexpr 1.02in+0\tabcolsep+0\arrayrulewidth}}{\fontsize{11}{11}\selectfont{\textcolor[HTML]{000000}{\global\setmainfont{Helvetica}{4}}}} & \multicolumn{1}{!{\color[HTML]{000000}\vrule width 0pt}>{\raggedright}p{\dimexpr 1.02in+0\tabcolsep+0\arrayrulewidth}!{\color[HTML]{000000}\vrule width 0pt}}{\fontsize{11}{11}\selectfont{\textcolor[HTML]{000000}{\global\setmainfont{Helvetica}{7}}}} \\





\multicolumn{1}{!{\color[HTML]{000000}\vrule width 0pt}>{\raggedright}p{\dimexpr 1.08in+0\tabcolsep+0\arrayrulewidth}}{\fontsize{11}{11}\selectfont{\textcolor[HTML]{000000}{\global\setmainfont{Helvetica}{Alder\ (år)}}}} & \multicolumn{1}{!{\color[HTML]{000000}\vrule width 0pt}>{\raggedright}p{\dimexpr 1.02in+0\tabcolsep+0\arrayrulewidth}}{\fontsize{11}{11}\selectfont{\textcolor[HTML]{000000}{\global\setmainfont{Helvetica}{24.5\ (1.29)}}}} & \multicolumn{1}{!{\color[HTML]{000000}\vrule width 0pt}>{\raggedright}p{\dimexpr 1.02in+0\tabcolsep+0\arrayrulewidth}!{\color[HTML]{000000}\vrule width 0pt}}{\fontsize{11}{11}\selectfont{\textcolor[HTML]{000000}{\global\setmainfont{Helvetica}{23.9\ (1.77)}}}} \\





\multicolumn{1}{!{\color[HTML]{000000}\vrule width 0pt}>{\raggedright}p{\dimexpr 1.08in+0\tabcolsep+0\arrayrulewidth}}{\fontsize{11}{11}\selectfont{\textcolor[HTML]{000000}{\global\setmainfont{Helvetica}{Vekt\ (kg)}}}} & \multicolumn{1}{!{\color[HTML]{000000}\vrule width 0pt}>{\raggedright}p{\dimexpr 1.02in+0\tabcolsep+0\arrayrulewidth}}{\fontsize{11}{11}\selectfont{\textcolor[HTML]{000000}{\global\setmainfont{Helvetica}{58.9\ (6.28)}}}} & \multicolumn{1}{!{\color[HTML]{000000}\vrule width 0pt}>{\raggedright}p{\dimexpr 1.02in+0\tabcolsep+0\arrayrulewidth}!{\color[HTML]{000000}\vrule width 0pt}}{\fontsize{11}{11}\selectfont{\textcolor[HTML]{000000}{\global\setmainfont{Helvetica}{74.8\ (5.55)}}}} \\





\multicolumn{1}{!{\color[HTML]{000000}\vrule width 0pt}>{\raggedright}p{\dimexpr 1.08in+0\tabcolsep+0\arrayrulewidth}}{\fontsize{11}{11}\selectfont{\textcolor[HTML]{000000}{\global\setmainfont{Helvetica}{Høyde\ (cm)}}}} & \multicolumn{1}{!{\color[HTML]{000000}\vrule width 0pt}>{\raggedright}p{\dimexpr 1.02in+0\tabcolsep+0\arrayrulewidth}}{\fontsize{11}{11}\selectfont{\textcolor[HTML]{000000}{\global\setmainfont{Helvetica}{166\ (2.99)}}}} & \multicolumn{1}{!{\color[HTML]{000000}\vrule width 0pt}>{\raggedright}p{\dimexpr 1.02in+0\tabcolsep+0\arrayrulewidth}!{\color[HTML]{000000}\vrule width 0pt}}{\fontsize{11}{11}\selectfont{\textcolor[HTML]{000000}{\global\setmainfont{Helvetica}{180\ (3.1)}}}} \\

\noalign{\global\setlength{\arrayrulewidth}{2pt}}\arrayrulecolor[HTML]{666666}\cline{1-3}



\end{longtable}

Figur 1 viser utviklingen Test 1 til Test 2 fordelt på kjønn. Det
typiske målefeilet (typical error, {[}@hopkins2000{]}) fra Test 1 til
Test 2 er utregnet til å være 4.04\%. Det maksimale oksygenopptaket til
de kvinnelige deltakerne spant seg fra 49.75 ml kg-1 min-1 til 54.50,
mens det hos de mannlige deltakerne spant seg fra 49.50 ml kg-1 min-1
til 62.05 ml kg-1 min-1.

\begin{figure}
\centering
\includegraphics{reliabilitet_files/figure-latex/Figur-1.pdf}
\caption{VO2max ved Test 1 og Test 2.}
\end{figure}

\hypertarget{diskusjon}{%
\subsection{Diskusjon}\label{diskusjon}}

Det typiske målefeilet(TE) på 4.04\% kan også tyde på at enkelte av
disse resultatene kan være utsatt for konfundering av ulik sort
{[}@hopkins2000{]}. TE på 4.04\% kan derfor tenkes å være et bilde
hvordan det kan se ut med få deltakere, med ulikt utgangspunkt, men også
uten skikkelig standardisering av treningshverdagen i forkant av
testene. Det kan også tenkes at med et varierende nivå hos deltagerne
kan enkelte oppleve en treningseffekt av Test 1. Samtidig som andre
kanskje ble slitne av å få en test inn i treningshverdagen.

Ettersom testing av maksimalt oksygenopptak er en test som gjennomføres
til utmattelse, vil man kunne forvente en viss variasjon i
testresultatene ettersom opplevd anstrengelse kan påvirkes av flere
ulike variabler {[}@halperin2015{]}. For å redusere systematisk skjehvet
i testresultat og målinger vil flere faktorer være nyttig å ta hensyn
til under slik testing. Som nevnt i metoden vil standardisering av
matinntak, koffeininntak, utstyr og tidspunkt for gjennomføring av test
være med på å kunne sikre intern validitet i resultatene. Eksempler er
deltakernes kjennskap til testen, verbal oppmuntring og personer
tilstede under testen er andre faktorer som potensielt kan bidra til å
påvirke resultatene. Felles for alle faktorer er at graden av påvirkning
på resultatene muligens reduseres ved hjelp av en standardisert
testprotokoll. Deltakerne - og testlederne, sin kjennskap til testen er
en annen faktor som trolig påvirker resultatene i vårt prosjekt. I dette
tilfellet fantes det enkelte deltakere som hadde gjennomført en liknende
test flere ganger, og en kan da forvente en mindre grad av variasjon
mellom resultatene på Test 1 og 2, sammenlignet med de deltakerne som
gjennomførte testen for første gang på Test 1. Dette fordi kjennskapen
og kunnskapen de tilegnet seg på Test 1, trolig spiller inn på
testresultatene.

Grunnen til at vi snakker om typisk målefeil(TE) er at når vi ønsker å
måle påvirkningen av trening på en gruppe individer er det viktig å
kunne si noe om hva som er endring og hva som er støy (målefeil). Desto
mindre støy en test innebærer jo bedre er målingen. Hva som danner denne
variasjonen som kan observeres gjennom TE er multifaktorelt, men
hoveddelen er som oftest biologisk {[}@hopkins2000{]}.

For å måle TE har vi brukt within subject deviation metoden. Denne
metoden påvirkes ikke av at gjennomsnittet endrer seg fra test til test
{[}@hopkins2000{]}. Data for målinger i VO2max fra fem sertifiserte
Australske laboratorier fastslo ett gjennomsnitt på 2.2\% for TE
{[}@halperin2015{]}. Data fra det Australske institutt for sport har
også fastslått at en TE på omtrent 2\% er riktig for både maksimal og
submaksimal O2 {[}@clark2007; @robertson2010; @saunders2009{]}. Dette
indikerer at med godt kalibrert utstyr og med utøvere som er godt vant
med testingen vil en TE på 2\% for det biologiske, og analytiske være
riktig {[}@halperin2015{]}.

\hypertarget{referanser}{%
\subsection{Referanser}\label{referanser}}

\end{document}
