% Options for packages loaded elsewhere
\PassOptionsToPackage{unicode}{hyperref}
\PassOptionsToPackage{hyphens}{url}
%
\documentclass[
]{article}
\usepackage{amsmath,amssymb}
\usepackage{lmodern}
\usepackage{ifxetex,ifluatex}
\ifnum 0\ifxetex 1\fi\ifluatex 1\fi=0 % if pdftex
  \usepackage[T1]{fontenc}
  \usepackage[utf8]{inputenc}
  \usepackage{textcomp} % provide euro and other symbols
\else % if luatex or xetex
  \usepackage{unicode-math}
  \defaultfontfeatures{Scale=MatchLowercase}
  \defaultfontfeatures[\rmfamily]{Ligatures=TeX,Scale=1}
\fi
% Use upquote if available, for straight quotes in verbatim environments
\IfFileExists{upquote.sty}{\usepackage{upquote}}{}
\IfFileExists{microtype.sty}{% use microtype if available
  \usepackage[]{microtype}
  \UseMicrotypeSet[protrusion]{basicmath} % disable protrusion for tt fonts
}{}
\makeatletter
\@ifundefined{KOMAClassName}{% if non-KOMA class
  \IfFileExists{parskip.sty}{%
    \usepackage{parskip}
  }{% else
    \setlength{\parindent}{0pt}
    \setlength{\parskip}{6pt plus 2pt minus 1pt}}
}{% if KOMA class
  \KOMAoptions{parskip=half}}
\makeatother
\usepackage{xcolor}
\IfFileExists{xurl.sty}{\usepackage{xurl}}{} % add URL line breaks if available
\IfFileExists{bookmark.sty}{\usepackage{bookmark}}{\usepackage{hyperref}}
\hypersetup{
  pdftitle={ARBEIDSKRAV 4},
  pdfauthor={Jacob Mollatt},
  hidelinks,
  pdfcreator={LaTeX via pandoc}}
\urlstyle{same} % disable monospaced font for URLs
\usepackage[margin=1in]{geometry}
\usepackage{longtable,booktabs,array}
\usepackage{calc} % for calculating minipage widths
% Correct order of tables after \paragraph or \subparagraph
\usepackage{etoolbox}
\makeatletter
\patchcmd\longtable{\par}{\if@noskipsec\mbox{}\fi\par}{}{}
\makeatother
% Allow footnotes in longtable head/foot
\IfFileExists{footnotehyper.sty}{\usepackage{footnotehyper}}{\usepackage{footnote}}
\makesavenoteenv{longtable}
\usepackage{graphicx}
\makeatletter
\def\maxwidth{\ifdim\Gin@nat@width>\linewidth\linewidth\else\Gin@nat@width\fi}
\def\maxheight{\ifdim\Gin@nat@height>\textheight\textheight\else\Gin@nat@height\fi}
\makeatother
% Scale images if necessary, so that they will not overflow the page
% margins by default, and it is still possible to overwrite the defaults
% using explicit options in \includegraphics[width, height, ...]{}
\setkeys{Gin}{width=\maxwidth,height=\maxheight,keepaspectratio}
% Set default figure placement to htbp
\makeatletter
\def\fps@figure{htbp}
\makeatother
\setlength{\emergencystretch}{3em} % prevent overfull lines
\providecommand{\tightlist}{%
  \setlength{\itemsep}{0pt}\setlength{\parskip}{0pt}}
\setcounter{secnumdepth}{-\maxdimen} % remove section numbering
\ifluatex
  \usepackage{selnolig}  % disable illegal ligatures
\fi
\newlength{\cslhangindent}
\setlength{\cslhangindent}{1.5em}
\newlength{\csllabelwidth}
\setlength{\csllabelwidth}{3em}
\newenvironment{CSLReferences}[2] % #1 hanging-ident, #2 entry spacing
 {% don't indent paragraphs
  \setlength{\parindent}{0pt}
  % turn on hanging indent if param 1 is 1
  \ifodd #1 \everypar{\setlength{\hangindent}{\cslhangindent}}\ignorespaces\fi
  % set entry spacing
  \ifnum #2 > 0
  \setlength{\parskip}{#2\baselineskip}
  \fi
 }%
 {}
\usepackage{calc}
\newcommand{\CSLBlock}[1]{#1\hfill\break}
\newcommand{\CSLLeftMargin}[1]{\parbox[t]{\csllabelwidth}{#1}}
\newcommand{\CSLRightInline}[1]{\parbox[t]{\linewidth - \csllabelwidth}{#1}\break}
\newcommand{\CSLIndent}[1]{\hspace{\cslhangindent}#1}

\title{ARBEIDSKRAV 4}
\author{Jacob Mollatt}
\date{11-11-2021}

\begin{document}
\maketitle

\hypertarget{studiedesign}{%
\section{Studiedesign}\label{studiedesign}}

\hypertarget{formuxe5l}{%
\subsection{Formål}\label{formuxe5l}}

Forfatterne i samtlige studier inkludert i denne oppgaven ønsker å bidra
til økt kunnskap som kan være med å løse den økende utfordringen med
hamstringsskader i idretten, ettersom denne utfordringen har store
konsekvenser for både prestasjon og økonomi (1--8). Forfatterne ønsker å
bidra til en økt kunnskap gjennom å formulere ulike spørsmål og
sammenligningsgrunnlag. Alle studier har som formål å sammenligne to
eller flere ulike rehabiliteringsprotokoller for å se hvilken av disse
som har best effekt på tid til retur til idrett. Mer spesifikt ønsker
tre av studiene å sammenligne hvilken effekt to ulike typer aktive
rehabiliteringsprotokoller har på tiden det tar for å returnere til
idrett etter en akutt hamstringsskade(4). De to andre studiene forsøker
å se hvilken effekt injeksjoner av Platelet-Rich-Plasma (PRP)- har på
rehabiliteringen av en hamstringsskade når det kombineres med aktiv
rehabilitering, målt gjennom tid til retur til idrett(6,7).

Den røde tråden i disse artiklene finner vi igjen i utgangspunktet
samtlige forfattere har i den tydelige byrden hamstringsskader har på
prestasjon og økonomi i toppidretten(1--3). Videre vil kunnskapen vi
allerede har, og dagens praksis omkring rehabiliteringen av
hamstringsskader være med å påvirke retningen på disse studiene (9--12).
Økt kunnskap leder ofte til økt nysgjerrighet, og tiltak for å ytterlige
optimalisere denne rehabiliteringsprosessen vil potensielt kunne gi
tydelige fortrinn i toppidretten på flere ulike måter, hvor prestasjon
og økonomi skiller seg ut som de to store.

\hypertarget{metodisk-sammenligning}{%
\subsection{Metodisk sammenligning}\label{metodisk-sammenligning}}

Detaljert karakteristikk i metoden til de ulike studiene inkludert
finnes i Tabell 1.

Alle de fem inkluderte artiklene er designet som randomiserte
kontrollerte studier(4--8). Deltakerne i studiene er hentet fra en
populasjon av voksne idrettsutøvere med akutte hamstringsskader. Tretten
ulike idretter er representert, med fotball og friidrett som de to
største. Utvalgsstørrelsen varierte fra 24 til 90, med et snitt på 62
deltakere over de fem studiene. Totalt var det 308 deltakere,
kjønnsfordelingen var omtrent 9\% kvinner og 91\% menn. To av studiene
hadde kun mannlige deltakere (4,6).Tre av studiene hadde på forhånd
gjort en utregning av ønsket utvalgsstørrelse for en statistisk styrke
på 80\% (4,6,7). Alle studiene sammenlignet ulike rehabiliteringsmetoder
av akutte hamstringsskader opp mot hverandre med retur til idrett som
utfallsmål, med mål om å finne forskjeller i utfallet av de ulike
rehabiliteringsmetodene. Retur til idrett var definert ulikt i flere av
studiene, og inklusjonskriteriene varierte også i stor grad mellom
studiene. Dette gjør det vanskeligere å sammenligne studiene opp mot
hverandre.

\textbf{Tabell 1 Studiekarakteristikk}

\begin{longtable}[]{@{}
  >{\raggedright\arraybackslash}p{(\columnwidth - 10\tabcolsep) * \real{0.11}}
  >{\raggedright\arraybackslash}p{(\columnwidth - 10\tabcolsep) * \real{0.15}}
  >{\raggedright\arraybackslash}p{(\columnwidth - 10\tabcolsep) * \real{0.15}}
  >{\raggedright\arraybackslash}p{(\columnwidth - 10\tabcolsep) * \real{0.21}}
  >{\raggedright\arraybackslash}p{(\columnwidth - 10\tabcolsep) * \real{0.17}}
  >{\raggedright\arraybackslash}p{(\columnwidth - 10\tabcolsep) * \real{0.19}}@{}}
\caption{\emph{*Hamstring strain injury - hamstringsskade\\
**Numerical Rating Scale\textbf{\hfill\break
}}***Platelet-Rich-Plasma\\
****Platelet-Poor-Plasma\\
}\tabularnewline
\toprule
Forf atter & Utvalg & Rekr uttering & Statistisk styrke & Inter
vensjoner & Utfallsmål \\
\midrule
\endfirsthead
\toprule
Forf atter & Utvalg & Rekr uttering & Statistisk styrke & Inter
vensjoner & Utfallsmål \\
\midrule
\endhead
Hi ckey, 2020 & 45 menn med akutt HSI* \textless7 dager.~ & Rekr
uttering skjedde løpende over en periode på 15mnd. Pot ensielle d
eltakere ble rekr uttering via reklame plakater og kontakt med lokale id
rettslag og kl inikker. D eltakere ble til slutt i nkludert på bakgrunn
av f orhåndsd efinerte inklus jonskrit erier.~~ & 29 deltakere var
ønsket for statistisk styrke på 80\%, medregnet en drop-out på 20\%.
Basert på en eff ektstørrelse på 1,2.~ & P rogressiv steg visrehabi
literings protokoll bestående av øvelser for å styrke h amstringm
uskulatur og løping. P rogresjon og gjen nomføring innenfor smer
tefrihet. 0 på NRS** & Tid til retur. Tid fra skad e-tidspunkt til
kriterier for retur til idrett var bestått.~ \\
~ & ~ & ~ & ~ & P rogressiv stegvis rehabi literings protokoll bestående
av øvelser for å styrke ha mstring-m uskulatur og løping. P rogresjon og
gjen nomføring innenfor en gitt sme rtegrense \textless4 på N RS**.~~ &
~ \\
& & & & & \\
& & & & & \\
Ask ling, 2013 & 75 fotball spillere med akutt HSI* \textless2 dager.~ &
Rekrut teringen skjedde over en løpende periode på 33 måneder. Gjennom
forf atternes idrettme disinske kontakt nettverk innen svensk fotball. D
eltakere ble til slutt i nkludert på bakgrunn av f orhåndsd efinerte
inklus jonskrit erier.~~ & Ikke definert~ & Rehabi literings protokoll
bestående av 3 øvelser med fokus på å styrke ha mstringen i hovedsak med
e ksentrisk mus kelarbeid (L-pr otokoll). & Tid til retur. Tid fra skad
e-tidspunkt til full deltakelse i lagstrening og tilg jengelighet for
kamp.~ \\
~ & ~ & ~ & ~ & Rehabi literings protokoll bestående av 3 øvelser med
fokus på å styrke ha mstringen (C-pr otokoll). & ~ \\
Ask ling, 2014 & 57 elite f riidrett sutøvere med akutt HSI* \textless2
dager. & Rekrut teringen skjedde over en løpende periode på 38 måneder.
Gjennom forf atternes idrettme disinske kontakt nettverk innen svensk fr
iidrett, og invi tasjoner ble også sendt ut gjennom det svenske frii
drettsfo rbundet. D eltakere ble til slutt i nkludert på bakgrunn av f
orhåndsd efinerte inklus jonskrit erier.~~ & Ikke definert~ & Rehabi
literings protokoll bestående av 3 øvelser med fokus på å styrke ha
mstringen i hovedsak med e ksentrisk mus kelarbeid (L-pr otokoll). & Tid
til retur. Tid fra skad e-tidspunkt til full deltakelse i trening. \\
~ & ~ & ~ & ~ & Rehabi literings protokoll bestående av 3 øvelser med
fokus på å styrke ha mstringen (C-pr otokoll). & ~ \\
H amid, 2014 & 24 Ik ke-profe sjonelle utøvere med akutt HSI* \textless7
dager & P asienter over 18 år som oppsøkte forf atternes I drettsme
disinske klinikk med mistenkt hamstri ngsskade ble u ndersøkt og i
nkludert etter diagnose og på bakgrunn og f orhåndsd efinerte inklu
sjonskri terier.~ & 28 deltakere (14 i hver intervens jons-gruppe) var
ønsket for statistisk styrke på 80\% med sig nifikansnivå satt til 0,05.
& Rehabi literings protokoll for å styrke hamstring samt p rogressiv
agili tytrening og styrke av kjernemu skulatur. Samt injeksjon av PRP
***.~~ & Tid til retur. Tid fra skad e-tidspunkt til kriterier for retur
til idrett var oppfylt. Målt ukentlig eller til studien avsluttet ble a
vsluttet(16 uker). \\
~ & ~ & ~ & ~ & Rehabi literings protokoll for å styrke hamstring samt p
rogressiv agili tytrening og styrke av kjernemu skulatur. & ~ \\
Hami lton, 2015 & 90 Idrett sutøvere henvist til ASPETAR med akutt HSI*
\textless5 dager. & P asienter ble re kruttert fra klubber og forbund ti
lknyttet et n asjonalt idrettsm edisinsk program i Qatar.M edisinsk
støtt eapparat henviste til studie senteret ved mistanke om en akutt
HSI. De ltakerne ble i nkludert på bakgrunn av f orhåndsd efinerte inklu
sjonskri terier.~ & 60 deltakere var ønsket for statistisk styrke på
80\% med sig nifikansnivå på 0,05 medregnet en drop-out på 10\%. &
Rehabi literings protokoll for å styrke hamstring (usual care), samt
injeksjon av PRP*** & Tid til retur. Tid fra skad e-tidspunkt til
gjennomført reh abilitering og kriterier for retur til idrett var
oppfylt.~~ \\
~ & ~ & ~ & ~ & Rehabi literings protokoll for å styrke hamstring (usual
care), samt injeksjon av placeb o-sammenl igner(PPP ****) & ~ \\
& & & & Rehabil iterings- protokoll for å styrke hamstring (usual care),
& \\
\bottomrule
\end{longtable}

\hypertarget{resultater}{%
\subsection{Resultater}\label{resultater}}

Statistiske analyser.

Hickey 2020

Intention-to-treat analyse ble brukt til å undersøke protokollens effekt
på antall dager til retur til idrett, og antallet re-skader, gjennom å
bruke en Cox--regresjonsmodell.Kaplan-Meier metode ble brukt for å
sammenligne kumulativ survival rate med tid til retur til idrett.
(survival package i R, versjon 3.4.3)

Askling 2013

Shapiro-Wilk W test ble brukt til å undersøke om data var normalfordelt,
noe de viste seg å ikke være. Mann-Withney U test ble brukt til å
undersøke forskjeller i deskriptive data, men også til å undersøke
forskjell i tid til retur til idrett mellom protokollene. Forfatterne
brukte også Spearman Rank order correlations for å undersøke hvilken
korrelasjon som fantes mellom tid til retur til idrett, MR-funn og
kliniske funn.~

Askling 2014

Shapiro-Wilk W test ble brukt til å undersøke om data var normalfordelt,
noe de viste seg å ikke være. Mann-Withney U test ble brukt til å
undersøke forskjeller i deskriptive data, men også til å undersøke
forskjell i tid til retur til idrett mellom protokollene. Forfatterne
brukte også Spearman Rank order correlations for å undersøke hvilken
korrelasjon som fantes mellom tid til retur til idrett, MR-funn og
kliniske funn

Hamid 2014

Kaplan-Meier metode ble brukt for å sammenligne kumulativ survival rate
med tid til retur til idrett. Cox regresjonsanalyse for å evaluere
effekt av intervensjon og andre covariater på tid til retur.

Hamilton 2015

Pearson Chi-squared test på kategoriske data og one-way ANOVA mellom
gruppene med kontinuerlige data. Kaplan-Meier metode ble brukt for å
sammenligne kumulativ survival rate med tid til retur til idrett. Cox
regresjonsmodell og linære regresjonsmodeller ble brukt for å se på
effekt av intervensjon på tid til retur til idrett.

\hypertarget{resultater-1}{%
\subsection{Resultater}\label{resultater-1}}

I studiene som sammenlignet to ulike aktive rehabiliteringsprotokoller
fant to av studiene signifikante forskjeller i tid til retur til idrett
(5). Den tredje av studiene fant ingen signifikant forskjell i tid til
retur til idrett(4). Blant de to studiene som undersøkte effekten av en
PRP injeksjon, fant den ene studien en signifikant kortere tid til retur
til idrett i gruppen med PRP injeksjon og standard rehabilitering,
sammenlignet med gruppen som kun gjennomgikk standard rehabilitering(7).
Den andre studien fant ingen signifikant forskjell i tiden til retur til
idrett mellom gruppen som mottok PRP-injeksjon og gruppen som mottok en
standard rehabilitering(6). Resultatene er oppsummert fordelt på
studiene under.

\hypertarget{hickey-2020-hickey2020}{%
\paragraph{Hickey, 2020 (4)}\label{hickey-2020-hickey2020}}

Ingen signifikant forskjell i hovedutfallsmål, som var tid til retur til
idrett. Sekundære utfallsmål på isometrisk knefleksjon og fasikkellengde
i biceps femoris caput longum var bedre i gruppen med rehabilitering
innenfor smertegrense 4 på Numerical Rating Scale(NRS). Median tid fra
skade til retur til idrett var 15 dager(95\% konfidensintervall) i
gruppen med smertefrihet, og 17 dager(95\% konfidensintervall) i gruppen
med smertegrense på 4 på NRS. P-verdi på 0.37.

\hypertarget{askling-2013-askling2013}{%
\paragraph{Askling, 2013 (8)}\label{askling-2013-askling2013}}

Signifikant kortere rehabilitering i gruppen med eksentrisk muskelarbeid
og forlenging av muskulatur(L-protokoll) sammenlignet med den andre
konvensjonelle rehabiliteringsgruppen(C-protokoll)(p\textless0.001).
Gjennomsnittlig tid til til retur til idrett i L-protokoll var 28 dager
(1SD±15, range 8-58 dager), sammenlignet med et snitt på 51 dager
(1SD±21, range 12-94 dager) i C-protokoll.

\hypertarget{askling-2014-askling2014}{%
\paragraph{Askling, 2014 (5)}\label{askling-2014-askling2014}}

Signifikant kortere rehabilitering i gruppen med eksentrisk muskelarbeid
og forlenging av muskulatur(L-protokoll) sammenlignet med den andre
konvensjonelle rehabiliteringsgruppen(C-protokoll)(p\textless0.001,
d=−1.21). Gjennomsnittlig tid til retur til idrett i L-protokoll var 49
dager (1SD±26, range 18--107 dager), sammenlignet med et snitt på 86
dager (1SD±34, range 26--140 dager) i C-protokoll.

\hypertarget{hamid-2014-ahamid2014}{%
\paragraph{Hamid, 2014 (7)}\label{hamid-2014-ahamid2014}}

Pasientene som mottok PRP injeksjon sammen med rehabiliteringsprogrammet
hadde en signifikant tidligere retur til idrett sammenlignet med de som
ikke mottok PRP-injeksjon(p = 0.02). Gjennomsnittlig tid til retur var
42.5±20.6 dager i kontrollgrupen, og 26.7±7 dager i PRP gruppen.

\hypertarget{hamilton2015-hamilton2015}{%
\paragraph{Hamilton,2015 (6)}\label{hamilton2015-hamilton2015}}

Justert differanse i tid til retur til idrett mellom PRP-gruppen og
PPP-gruppen var −5.7 dager (95\% CI, p=0.01). Mellom PRP og
kontrollgruppen −2.9 dager (95\% CI, p=0.189) og mellom PPP og
kontrollgruppen 2.8 dager (95\% CI, p=0.210).

\hypertarget{konklusjoner}{%
\subsection{Konklusjoner}\label{konklusjoner}}

Tre av studiene konkluderte med at de fant en forskjell i effekten de
ulike intervensjonene de undersøkte hadde på tid til retur til
idrett(5,7,8). To av disse studiene konkluderte med at en
rehabiliteringsprotokoll med eksentrisk muskelarbeid og overvekt av
forlenging av muskulatur i muskelarbeidet er mer effektivt en
konvensjonelle rehabiliteringsøvelser(5). Den siste studien konkluderte
med at en PRP-injeksjon kombinert med standard rehabilitering hadde
større positiv effekt på tid til retur til idrett enn en standard
rehabilitering alene(7). De to siste studiene konkluderte med at det
ikke var noen forskjell i effekten de ulike intervensjonene de
undersøkte hadde på tid til retur til idrett(6). To av studiene
inkludert kom til motsvarende konklusjoner, der den ene konkluderte med
at en enkelt PRP-injeksjon ikke hadde effekt på tid til retur til
idrett, mens den andre konkluderte med at en enkelt PRP-injeksjon hadde
positiv effekt på tid til retur til idrett(7). Konklusjoner er
oppsummert og fordelt på de ulike studiene under.

\hypertarget{hickey-2020-hickey2020-1}{%
\paragraph{Hickey, 2020 (4)}\label{hickey-2020-hickey2020-1}}

Rehabilitering innenfor smertegrense på 4 på NRS ga ingen raskere retur
til idrett enn rehabilitering innenfor smertefrihet, men bedret andre
utfallsmål som isometrisk styrke i knefleksjon og fasikkellengde.

\hypertarget{askling-2013-askling2013-1}{%
\paragraph{Askling, 2013 (8)}\label{askling-2013-askling2013-1}}

En rehabiliteringsprotokoll med eksentrisk muskelarbeid og overvekt av
forlenging av muskulatur i muskelarbeidet er mer effektiv enn en
protokoll med konvensjonelle rehabiliteringsøvelser for å sikre en
raskest mulig retur til idrett i svensk toppfotball.

\hypertarget{askling-2014-askling2014-1}{%
\paragraph{Askling, 2014 (5)}\label{askling-2014-askling2014-1}}

En rehabiliteringsprotokoll med eksentrisk muskelarbeid og overvekt av
forlenging av muskulatur i muskelarbeidet er mer effektiv enn en
protokoll med konvensjonelle rehabiliteringsøvelser for å sikre en
raskest mulig retur til idrett hos svenske friidrettsutøvere.

\hypertarget{hamid-2014-ahamid2014-1}{%
\paragraph{Hamid, 2014 (7)}\label{hamid-2014-ahamid2014-1}}

En enkelt PRP injeksjon kombinert med et standard
rehabiliteringsprogramm ga signifikant mer effektiv behandling av
hamstringsskader enn standard rehabilitering alene.

\hypertarget{hamilton2015-hamilton2015-1}{%
\paragraph{Hamilton,2015 (6)}\label{hamilton2015-hamilton2015-1}}

Det er ingen indikasjon på at det er tydelige fordeler ved en enkelt
PRP-injeksjon sammenlignet med en intensiv rehabiliteringsprotokoll hos
utøvere som har gjennomgått en akutt hamstringsskade.

\hypertarget{referanser}{%
\subsection*{Referanser}\label{referanser}}
\addcontentsline{toc}{subsection}{Referanser}

\hypertarget{refs}{}
\begin{CSLReferences}{0}{0}
\leavevmode\hypertarget{ref-ekstrand2011}{}%
\CSLLeftMargin{1. }
\CSLRightInline{Ekstrand J, Hägglund M, Waldén M. Epidemiology of Muscle
Injuries in Professional Football (Soccer). The American Journal of
Sports Medicine {[}Internet{]}. 2011 Jun;39(6):1226--32. Available from:
\url{http://journals.sagepub.com/doi/10.1177/0363546510395879}}

\leavevmode\hypertarget{ref-hickey2014}{}%
\CSLLeftMargin{2. }
\CSLRightInline{Hickey J, Shield AJ, Williams MD, Opar DA. The financial
cost of hamstring strain injuries in the Australian Football League.
British Journal of Sports Medicine. 2014 Apr;48(8):729--30. }

\leavevmode\hypertarget{ref-eirale2013}{}%
\CSLLeftMargin{3. }
\CSLRightInline{Eirale C, Tol JL, Farooq A, Smiley F, Chalabi H. Low
injury rate strongly correlates with team success in Qatari professional
football. British Journal of Sports Medicine. 2013 Aug;47(12):807--8. }

\leavevmode\hypertarget{ref-hickey2020}{}%
\CSLLeftMargin{4. }
\CSLRightInline{Hickey JT, Timmins RG, Maniar N, Rio E, Hickey PF,
Pitcher CA, et al. Pain-Free Versus Pain-Threshold Rehabilitation
Following Acute Hamstring Strain Injury: A Randomized Controlled Trial.
Journal of Orthopaedic \& Sports Physical Therapy {[}Internet{]}. 2020
Feb;50(2):91--103. Available from:
\url{http://dx.doi.org/10.2519/jospt.2020.8895}}

\leavevmode\hypertarget{ref-askling2014}{}%
\CSLLeftMargin{5. }
\CSLRightInline{Askling CM, Tengvar M, Tarassova O, Thorstensson A.
Acute hamstring injuries in Swedish elite sprinters and jumpers: a
prospective randomised controlled clinical trial comparing two
rehabilitation protocols. British Journal of Sports Medicine. 2014
Apr;48(7):532--9. }

\leavevmode\hypertarget{ref-hamilton2015}{}%
\CSLLeftMargin{6. }
\CSLRightInline{Hamilton B, Tol JL, Almusa E, Boukarroum S, Eirale C,
Farooq A, et al. Platelet-rich plasma does not enhance return to play in
hamstring injuries: a randomised controlled trial. British Journal of
Sports Medicine {[}Internet{]}. 2015 Jul;49(14):943--50. Available from:
\url{http://dx.doi.org/10.1136/bjsports-2015-094603}}

\leavevmode\hypertarget{ref-ahamid2014}{}%
\CSLLeftMargin{7. }
\CSLRightInline{A Hamid MS, Mohamed Ali MR, Yusof A, George J, Lee LPC.
Platelet-rich plasma injections for the treatment of hamstring injuries:
a randomized controlled trial. The American Journal of Sports Medicine.
2014 Oct;42(10):2410--8. }

\leavevmode\hypertarget{ref-askling2013}{}%
\CSLLeftMargin{8. }
\CSLRightInline{Askling CM, Tengvar M, Thorstensson A. Acute hamstring
injuries in Swedish elite football: a prospective randomised controlled
clinical trial comparing two rehabilitation protocols. British journal
of sports medicine. 2013 Oct;47(15):953--9. }

\leavevmode\hypertarget{ref-brooks2006}{}%
\CSLLeftMargin{9. }
\CSLRightInline{Brooks JHM, Fuller CW, Kemp SPT, Reddin DB. Incidence,
risk, and prevention of hamstring muscle injuries in professional rugby
union. The American Journal of Sports Medicine. 2006
Aug;34(8):1297--306. }

\leavevmode\hypertarget{ref-horst2017}{}%
\CSLLeftMargin{10. }
\CSLRightInline{Horst N van der, Backx FJG, Goedhart EA, Huisstede BM.
Return to play after hamstring injuries in football (soccer): a
worldwide Delphi procedure regarding definition, medical criteria and
decision-making. Br J Sports Med {[}Internet{]}. 2017 Nov
1;51(22):1583--91. Available from:
\url{https://bjsm.bmj.com/content/51/22/1583}}

\leavevmode\hypertarget{ref-opar2012}{}%
\CSLLeftMargin{11. }
\CSLRightInline{Opar DA, Williams MD, Shield AJ. Hamstring strain
injuries: factors that lead to injury and re-injury. Sports Medicine
(Auckland, NZ). 2012 Mar 1;42(3):209--26. }

\leavevmode\hypertarget{ref-podlog2014}{}%
\CSLLeftMargin{12. }
\CSLRightInline{Podlog L, Heil J, Schulte S. Psychosocial factors in
sports injury rehabilitation and return to play. Physical Medicine and
Rehabilitation Clinics of North America. 2014 Nov;25(4):915--30. }

\end{CSLReferences}

\end{document}
